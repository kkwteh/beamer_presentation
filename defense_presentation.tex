\documentclass{beamer}
\usetheme{Marburg}

\usepackage{ esint }

\def\bA{{\mathbb A}}
\def\bB{{\mathbb B}}
\def\bC{{\mathbb C}}
\def\bD{{\mathbb D}}
\def\bE{{\mathbb E}}
\def\bF{{\mathbb F}}
\def\bG{{\mathbb G}}
\def\bH{{\mathbb H}}
\def\bI{{\mathbb I}}
\def\bJ{{\mathbb J}}
\def\bK{{\mathbb K}}
\def\bL{{\mathbb L}}
\def\bM{{\mathbb M}}
\def\bN{{\mathbb N}}
\def\bO{{\mathbb O}}
\def\bP{{\mathbb P}}
\def\bQ{{\mathbb Q}}
\def\bR{{\mathbb R}}
\def\bS{{\mathbb S}}
\def\bT{{\mathbb T}}
\def\bU{{\mathbb U}}
\def\bV{{\mathbb V}}
\def\bW{{\mathbb W}}
\def\bX{{\mathbb X}}
\def\bY{{\mathbb Y}}
\def\bZ{{\mathbb Z}}

\def\A{{\mathbb A}}
\def\C{{\mathbb C}}
\def\F{{\mathbb F}}
\renewcommand{\H}{{\mathbb H}}
\def\N{{\mathbb N}}
\renewcommand{\P}{{\mathbb P}}
\def\Q{{\mathbb Q}}
\def\Z{{\mathbb Z}}
\def\R{{\mathbb R}}
\def\K{{\mathbb K}}

\def\cA{{\mathcal A}}
\def\cB{{\mathcal B}}
\def\cC{{\mathcal C}}
\def\cD{{\mathcal D}}
\def\cE{{\mathcal E}}
\def\cF{{\mathcal F}}
\def\cG{{\mathcal G}}
\def\cH{{\mathcal H}}
\def\cI{{\mathcal I}}
\def\cJ{{\mathcal J}}
\def\cK{{\mathcal K}}
\def\cL{{\mathcal L}}
\def\cM{{\mathcal M}}
\def\cN{{\mathcal N}}
\def\cO{{\mathcal O}}
\def\cP{{\mathcal P}}
\def\cQ{{\mathcal Q}}
\def\cR{{\mathcal R}}
\def\cS{{\mathcal S}}
\def\cT{{\mathcal T}}
\def\cU{{\mathcal U}}
\def\cV{{\mathcal V}}
\def\cW{{\mathcal W}}
\def\cX{{\mathcal X}}
\def\cY{{\mathcal Y}}
\def\cZ{{\mathcal Z}}

\def\Aut{{\rm Aut}}
\def\Coker{{\rm Coker}}
\def\Diff{{\rm Diff}}
\def\dist{{\rm dist}}
\def\Dom{{\rm Dom}}
\def\Difg{{\rm Difg}}
\def\End{{\rm End}}
\def\Ext{{\rm Ext}}
\def\Gal{{\rm Gal}}
\def\GL{{\rm GL}}
\def\Gr{{\rm Gr}}
\def\Hom{{\rm Hom}}
\def\id{{\rm id}}
\def\Ind{{\rm Ind}}
\def\Index{{\rm Ind}}
\def\Inf{{\rm Inf}}
\def\Inn{{\rm Inn}}
\def\Int{{\rm Int}}
\def\Ker{{\rm Ker}}
\def\Lie{{\rm Lie}}
\def\Li{{\rm Li}}
\def\Lim{{\rm Lim}}
\def\Mod{{\rm Mod}}
\def\Out{{\rm Out}}
\def\PGL{{\rm PGL}}
\def\PSL{{\rm PSL}}
\def\rank{{\rm rank}}
\def\Res{{\rm Res}}
\def\Rep{{\rm Rep}}
\def\sign{{\rm sign}}
\def\SL{{\rm SL}}
\def\Spec{{\rm Spec}}
\def\Sp{{\rm Spec}}
\def\spin{{\rm spin}}
\def\Spin{{\rm Spin}}
\def\SU{{\rm SU}}
\def\Sup{{\rm Sup}}
\def\Trace{{\rm Tr}}
\def\Tr{{\rm Tr}}
\def\tr{{\rm tr}}
\def\Vect{{\rm Vect}}

\def\elel{(  \downarrow 1)}
\def\dodo{(   \downarrow 3)}
\def\nunu{(   \uparrow 1)}
\def\upup{(   \uparrow 3)}
\def\mass{Y}

\def\fa{{\mathfrak a}}
\def\fb{{\mathfrak b}}
\def\fc{{\mathfrak c}}
\def\fd{{\mathfrak d}}
\def\fe{{\mathfrak e}}



\title{Dirac Spectra, Summation Formulae, and the Spectral Action} \author{Kevin Teh} \date{May 17, 2013}


\makeatletter
  \setbeamertemplate{sidebar \beamer@sidebarside}
  {
    \beamer@tempdim=\beamer@sidebarwidth%
    \advance\beamer@tempdim by -6pt%
    \vskip4em%
    \insertverticalnavigation{\beamer@sidebarwidth}%
    \vfill
    \ifx\beamer@sidebarside\beamer@lefttext%
    \else%
      \usebeamercolor{normal text}%
      \llap{\usebeamertemplate***{navigation symbols}\hskip0.1cm}%
      \vskip2pt%
    \fi%
  }%

  \ifx\beamer@sidebarside\beamer@lefttext%
    \defbeamertemplate*{sidebar right}{sidebar theme}
    {%
      \vfill%
      \llap{\usebeamertemplate***{navigation symbols}\hskip0.1cm}%
      \vskip2pt}
  \fi

\setbeamertemplate{section in sidebar}%{sidebar theme}
{%
  \vbox{%
    \vskip1ex%
    \beamer@sidebarformat{3pt}{section in sidebar}{\insertsectionheadnumber
~\insertsectionhead}%
  }%
}
\setbeamertemplate{section in sidebar shaded}%{sidebar theme}
{%
  \vbox{%
    \vskip1ex%
    \beamer@sidebarformat{3pt}{section in sidebar shaded}{\insertsectionheadnumber
~\insertsectionhead}%
  }%
}
\makeatother

\title{Dirac Spectra, Summation Formulae, and the Spectral Action}
\author{Kevin Teh}
\institute{Caltech}
\date{May 17, 2013}

\begin{document}

\begin{frame}
\titlepage
\end{frame}

\section{Intro to NCG and spectral action}

\subsection{Spectral Triples}
\begin{frame}
	\frametitle{Spectral Triples - Definition}
	\begin{block}{Definition}
		A spectral triple $(\cA, \cH, \cD)$ is given by an involutive unital algebra $\cA$ represented as operators in a Hilbert space $\cH$ and a self-adjoint operators $\cD$ with compact resolvent such that all commutators $[\cD, a]$ are bounded for $a \in \cA$.
	\end{block}
	\pause

	\begin{block}{}
		A spectral triple is even if the Hilbert space $\cH$ is equipped with a $\Z /2 $-grading $\gamma$ which commutes with any $a \in \cA$ and anti-commutes with $\cD$.
\end{block}
\end{frame}


\begin{frame}
	\frametitle{Spectral Triples - Example}
	\begin{exampleblock}{Compact spin manifold $M$}
		\begin{itemize}
			\item $\cA = C^{\infty}(M)$
			\item $\cH = L^2(\Gamma, S)$, square-integrable sections of spinor bundle
			\item $\cD$ Dirac operator associated with spin structure
		\end{itemize}
	\end{exampleblock}
\end{frame}

\subsection{Spectral Action}

\begin{frame}
	\frametitle{Spectral Action}

	\begin{block}{}
		For a spectral triple, $(\cA, \cH, \cD)$, the spectral action is given by
		\[
		\Tr f(\cD / \Lambda)
		\]
	\end{block}

	\pause

	\begin{itemize}
		\item $f \in \mathcal{S} (\R)$
		\item $\Lambda > 0$
	\end{itemize}

	\pause

	\begin{block}{}
		The spectral action plays a central role in the Noncommutative Standard Model
	\end{block}

\end{frame}


\subsection{Noncommutative Standard Model}

\begin{frame}
	\frametitle{Noncommutative Standard Model}

	\begin{block}{}
		A principal application of noncommutative geometry is a formulation of the standard model of physics using the mathematical framework of noncommutative geometry.
	\end{block}

	\pause

	\begin{block}{}
		Simple input for the model: the finite-dimensional algebra $\C \oplus \H \oplus \H \oplus M_3(\C)$
	\end{block}

	\pause

	\begin{block}{}
		Spacetime is modeled as the product geometry of a finite-dimensional spectral triple of the above algebra with a 4-dimensional compact Riemannian spin manifold and one may recover the usual standard model
	\end{block}

\end{frame}


\begin{frame}
	\frametitle{Noncommutative Standard Model}
	\begin{block}{}
		By considering ``inner fluctuations'' of the Dirac operator, one obtains the gauge bosons of the theory.
	\end{block}
\end{frame}

\section{Perturbative vs Non-perturbative}

\begin{frame}
	\frametitle{Perturbative vs Non-perturbative}
	\begin{block}{Perturbative expansion of spectral action via Gilkey's Theorem}
		\[
			\Tr f (D/\Lambda) \sim 2 \Lambda^4 f_4 a_0 + 2 \Lambda^2 f_2 a_2 + f_0 a_4 + \cdots + \Lambda^{-2k} f_{-2k} a_{4+2k} + \cdots
		\]

		\begin{align*}
			f_k &= \int_{0}^{\infty} f(u) u^{k-1} du, ~ k>0 \\
			&= f(0), ~ k = 0\\
			&= (-1)^{-k/2} \frac{(-k/2)!}{-k!}f^{(-k)}(0), ~ k < 0
		\end{align*}
	\end{block}

	\pause

	\begin{alertblock}{This is only an asymptotic series}

	\end{alertblock}
\end{frame}
\end{document}

\documentclass{beamer}
\usetheme{Malmoe}

\usepackage{ esint }

\def\bA{{\mathbb A}}
\def\bB{{\mathbb B}}
\def\bC{{\mathbb C}}
\def\bD{{\mathbb D}}
\def\bE{{\mathbb E}}
\def\bF{{\mathbb F}}
\def\bG{{\mathbb G}}
\def\bH{{\mathbb H}}
\def\bI{{\mathbb I}}
\def\bJ{{\mathbb J}}
\def\bK{{\mathbb K}}
\def\bL{{\mathbb L}}
\def\bM{{\mathbb M}}
\def\bN{{\mathbb N}}
\def\bO{{\mathbb O}}
\def\bP{{\mathbb P}}
\def\bQ{{\mathbb Q}}
\def\bR{{\mathbb R}}
\def\bS{{\mathbb S}}
\def\bT{{\mathbb T}}
\def\bU{{\mathbb U}}
\def\bV{{\mathbb V}}
\def\bW{{\mathbb W}}
\def\bX{{\mathbb X}}
\def\bY{{\mathbb Y}}
\def\bZ{{\mathbb Z}}

\def\A{{\mathbb A}}
\def\C{{\mathbb C}}
\def\F{{\mathbb F}}
\renewcommand{\H}{{\mathbb H}}
\def\N{{\mathbb N}}
\renewcommand{\P}{{\mathbb P}}
\def\Q{{\mathbb Q}}
\def\Z{{\mathbb Z}}
\def\R{{\mathbb R}}
\def\K{{\mathbb K}}

\def\cA{{\mathcal A}}
\def\cB{{\mathcal B}}
\def\cC{{\mathcal C}}
\def\cD{{\mathcal D}}
\def\cE{{\mathcal E}}
\def\cF{{\mathcal F}}
\def\cG{{\mathcal G}}
\def\cH{{\mathcal H}}
\def\cI{{\mathcal I}}
\def\cJ{{\mathcal J}}
\def\cK{{\mathcal K}}
\def\cL{{\mathcal L}}
\def\cM{{\mathcal M}}
\def\cN{{\mathcal N}}
\def\cO{{\mathcal O}}
\def\cP{{\mathcal P}}
\def\cQ{{\mathcal Q}}
\def\cR{{\mathcal R}}
\def\cS{{\mathcal S}}
\def\cT{{\mathcal T}}
\def\cU{{\mathcal U}}
\def\cV{{\mathcal V}}
\def\cW{{\mathcal W}}
\def\cX{{\mathcal X}}
\def\cY{{\mathcal Y}}
\def\cZ{{\mathcal Z}}

\def\Aut{{\rm Aut}}
\def\Cas{{\rm Cas}}
\def\Coker{{\rm Coker}}
\def\Diff{{\rm Diff}}
\def\dist{{\rm dist}}
\def\Dom{{\rm Dom}}
\def\Difg{{\rm Difg}}
\def\End{{\rm End}}
\def\Ext{{\rm Ext}}
\def\Gal{{\rm Gal}}
\def\GL{{\rm GL}}
\def\Gr{{\rm Gr}}
\def\Hom{{\rm Hom}}
\def\id{{\rm id}}
\def\Ind{{\rm Ind}}
\def\Index{{\rm Ind}}
\def\Inf{{\rm Inf}}
\def\Inn{{\rm Inn}}
\def\Int{{\rm Int}}
\def\Ker{{\rm Ker}}
\def\Lie{{\rm Lie}}
\def\Li{{\rm Li}}
\def\Lim{{\rm Lim}}
\def\Mod{{\rm Mod}}
\def\Out{{\rm Out}}
\def\PGL{{\rm PGL}}
\def\PSL{{\rm PSL}}
\def\rank{{\rm rank}}
\def\Res{{\rm Res}}
\def\Rep{{\rm Rep}}
\def\sign{{\rm sign}}
\def\SL{{\rm SL}}
\def\Spec{{\rm Spec}}
\def\Sp{{\rm Spec}}
\def\spin{{\rm spin}}
\def\Spin{{\rm Spin}}
\def\SU{{\rm SU}}
\def\Sup{{\rm Sup}}
\def\Trace{{\rm Tr}}
\def\Tr{{\rm Tr}}
\def\tr{{\rm tr}}
\def\Vect{{\rm Vect}}

\def\elel{(  \downarrow 1)}
\def\dodo{(   \downarrow 3)}
\def\nunu{(   \uparrow 1)}
\def\upup{(   \uparrow 3)}
\def\mass{Y}

\def\fa{{\mathfrak a}}
\def\fb{{\mathfrak b}}
\def\fc{{\mathfrak c}}
\def\fd{{\mathfrak d}}
\def\fe{{\mathfrak e}}



\title{Dirac Spectra, Summation Formulae, and the Spectral Action} \author{Kevin Teh} \date{May 17, 2013}



\title{Dirac Spectra, Summation Formulae, and the Spectral Action}
\author{Kevin Teh}
\institute{Caltech}
\date{May 17, 2013}

\begin{document}

\begin{frame}
\titlepage
\end{frame}

\section{Intro to NCG and spectral action}

\subsection{Spectral Triples}
\begin{frame}
  \frametitle{Spectral Triples - Definition}
  \begin{block}{Definition}
    A spectral triple $(\cA, \cH, D)$ is given by an involutive unital algebra $\cA$ represented as operators in a Hilbert space $\cH$ and a self-adjoint operators $D$ with compact resolvent such that all commutators $[D, a]$ are bounded for $a \in \cA$.
  \end{block}
  \pause

  \begin{block}{}
    A spectral triple is even if the Hilbert space $\cH$ is equipped with a $\Z /2 $-grading $\gamma$ which commutes with any $a \in \cA$ and anti-commutes with $D$.
\end{block}
\end{frame}


\begin{frame}
  \frametitle{Spectral Triples - Example}
  \begin{exampleblock}{Compact spin manifold $M$}
    \begin{itemize}
      \item $\cA = C^{\infty}(M)$
      \item $\cH = L^2(M, S)$, square-integrable sections of spinor bundle
      \item $D$ Dirac operator associated with spin structure
    \end{itemize}
  \end{exampleblock}
\end{frame}

\subsection{Spectral Action}

\begin{frame}
  \frametitle{Spectral Action}

  \begin{block}{}
    For a spectral triple, $(\cA, \cH, D)$, the spectral action is given by
    \[
    \Tr f(D / \Lambda)
    \]
  \end{block}

  \pause

  \begin{itemize}
    \item Typically, $f$ is a smooth approximation to a cutoff function.
    \item For our purposes, $f$ is an even function, $f \in \mathcal{S} (\R)$
    \item $\Lambda > 0$
    \item It is equivalent to calculate $f(D^2/ \Lambda ^2)$, since we may regard this simply as a different choice of the function $f$.
  \end{itemize}

  \pause

  \begin{block}{}
    The spectral action plays a central role in the Noncommutative Standard Model
  \end{block}

\end{frame}


\subsection{Noncommutative Standard Model}

\begin{frame}
  \frametitle{Noncommutative Standard Model}

  \begin{block}{}
    A principal application of noncommutative geometry is a formulation of the standard model of physics using the mathematical framework of noncommutative geometry.
  \end{block}

  \pause

  \begin{block}{}
    Simple input for the model: the finite-dimensional algebra $\C \oplus \H \oplus \H \oplus M_3(\C)$
  \end{block}

  \pause

  \begin{block}{}
    Spacetime is modeled as the product geometry of a finite-dimensional spectral triple of the above algebra with a 4-dimensional compact Riemannian spin manifold and one may recover the usual standard model
  \end{block}

\end{frame}


\begin{frame}
  \frametitle{Noncommutative Standard Model and Spectral Action}
  \begin{block}{}
    By computing the spectral action for ``inner fluctuations'' of the Dirac operator, one obtains the gauge bosons of the theory.
  \end{block}

  \begin{block}{}
    To give a flavour of what this involves, one must first equip the geometry with a real structure, $J$ (an antilinear unitary operator). Inner fluctuations of the Dirac operator are given by
    \[
    D_A = D + A + J A J^{-1},~ A = \sum a_j [D, b_j],~ a_j, b_j \in \cA,~ A = A^*
    \]
  \end{block}
\end{frame}

\subsection{Noncommutative Cosmology}

\section{Spectral Action}

\subsection{Perturbative vs Non-perturbative}

\begin{frame}
  \frametitle{Perturbative vs Non-perturbative}
  \begin{block}{Perturbative expansion of spectral action via Gilkey's Theorem}
    \[
      \Tr f (D/\Lambda) \sim 2 \Lambda^4 f_4 a_0 + 2 \Lambda^2 f_2 a_2 + f_0 a_4 + \cdots + \Lambda^{-2k} f_{-2k} a_{4+2k} + \cdots
    \]

    \begin{align*}
      f_k &= \int_{0}^{\infty} f(u) u^{k-1} du, ~ k>0 \\
      &= f(0), ~ k = 0\\
      &= (-1)^{-k/2} \frac{(-k/2)!}{-k!}f^{(-k)}(0), ~ k < 0
    \end{align*}
  \end{block}

  \pause

  \begin{alertblock}{This is only an asymptotic series}

  \end{alertblock}
\end{frame}

\begin{frame}
  \frametitle{Perturbative vs Non-perturbative}
  \begin{block}{Nonperturbative calculation of the spectral action}
    \[
      \Tr f (D/\Lambda) = \sum_{n \in \Z} h(n) = \sum_{n \in \Z} \widehat{h}(n)
    \]
  \end{block}

  \pause

  \begin{block}{}
    Uses the Poisson summation formula

    Returns a convergent series
  \end{block}
\end{frame}

\begin{frame}
  \frametitle{The First Example: $SU(2)$}
  \begin{block}{Usual Expression of Dirac Spectrum}
    $\rm{Spec}(D) = \{ \pm (\frac{3}{2} + k) | k \in \Z, k \geq 0\}$

    multiplicity of $(\frac{3}{2} + k)$ is $2 {k+2} \choose {k}$
  \end{block}

  \pause

  \begin{block}{Alternative Expression}
    $\rm{Spec}(D) = \{n + 1/2 | n \in \Z \}$

    multiplicity of $n + 1/2$ is $n(n+1)$
  \end{block}

  \pause

  \begin{block}{}
    Important features of alternative expression:
    \begin{itemize}
      \item spectrum is indexed by $\Z$
      \item multiplicity is a polynomial, and in particular, smooth
    \end{itemize}
  \end{block}
\end{frame}

\begin{frame}
  \frametitle{The First Example: $SU(2)$ aka $S^3$}
  \begin{block}{}
    Define $g(u) = (u^2 -1/4)f(u/\Lambda)$
  \end{block}

  \pause

  \begin{block}{}
    \begin{align*}
      \widehat{g}(x) &= \int_{\R}g(u) e^{-2 \pi i x u}du = \int_{\R}(u^2 - \frac{1}{4})f(u/\Lambda)e^{-2 \pi i x u}du \\
      &= \Lambda ^3 \int_{\R}v^2 f(v) e^{-2\pi i \Lambda x v}dv - \frac{1}{4} \Lambda \int_{\R}f(v)e^{-2\pi i \Lambda xv}dv
    \end{align*}
  \end{block}

  \pause

  \begin{block}{}
    Applying the Poisson summation formula to $g$ yields

    \[
      Tr(f(D/\Lambda)) = \Lambda^3 \sum_{\Z}(-1)^n \widehat{f}^{(2)}(\Lambda n) - \frac{1}{4} \Lambda \sum_{\Z}(-1)^n \widehat{f}(\Lambda n)
    \]
  \end{block}
\end{frame}

\begin{frame}
  \frametitle{The First Example: $SU(2)$ aka $S^3$}
  \begin{block}{}
    Since we took $f \in \cS (\R)$, we have the estimates

    \[
      \sum_{n \neq 0}|\widehat{f}(\Lambda n)| \leq C_k \Lambda^{-k}, ~ \sum_{n \neq 0}|\widehat{f}^{(2)}(\Lambda n)| \leq C_k \Lambda^{-k}
    \]
  \end{block}

  \pause

  \begin{block}{}
    This gives the following theorem (Chamseddine, Connes)

    \[
      Tr(f(D/\Lambda)) = \Lambda^3 \int_\R v^2 f(v) dv - \frac{1}{4}\Lambda \int_{\R}f(v)dv + O(\Lambda)^{-k}
    \]
  \end{block}

  \pause

  \begin{block}{}
    Concrete estimates of the remainder are possible
  \end{block}
\end{frame}

\begin{frame}
  \begin{block}{}
  We want to know: Are there other situations where this technique works?
  \end{block}
\end{frame}

\begin{frame}
  \frametitle{Nonexample: $S^4$}
  \begin{block}{Spectrum}
    $\Spec(D) = \{\pm (2 + k) | k \in \Z k \geq 0 \}$

    $\rm{multiplicity} = 4 {{k+3} \choose {k}}$
  \end{block}

  \pause

  \begin{alertblock}{The technique doesn't work}
    When one tries to parametrize the spectrum by $\Z$, the absolute value $|x|$ appears in the expression for the multiplicity. The resulting function is no longer smooth, and its Fourier transform no longer has the nice property of being in $\cS$
  \end{alertblock}
\end{frame}

\begin{frame}
  \frametitle{Spaces that we will consider}
  \begin{itemize}
    \item Coset spaces of $SU(2)$
    \item Bieberbach manifolds (Quotients of flat $\mathbb{T} ^3$)
    \item Coset spaces of $SU(2)$ with matter
    \item $SU(3)$
  \end{itemize}

\end{frame}

\section{Coset spaces of $SU(2)$}

\begin{frame}
  \frametitle{Coset spaces of $SU(2)$}
  \begin{block}{}
    The spaces we are considering are the Riemannian homogeneous spaces $SU(2)/\Gamma$ where $\Gamma$ is a finite subgroup of $SU(2)$.

    Each of the following groups are subgroups of $SU(2)$ and are unique up to conjugacy:
    \begin{itemize}
      \item cyclic group of order $N$, $N = 1,2,3,\ldots$
      \item dicyclic group of order $4N$, $N = 2,3,\ldots$
      \item binary tetrahedral group
      \item binary octahedral group
      \item binary icosahedral group
    \end{itemize}
  \end{block}
\end{frame}

\begin{frame}
  \frametitle{Coset spaces of $SU(2)$ - Spectrum}
  \begin{block}{}
    The Hilbert space of $L^2$-spinors of $G/H$ may be given the representation theoretic form

    \[
      \overline{\oplus_{\gamma \in \hat{G}} V_{\gamma} \otimes \Hom_{H}(V_{\gamma}, \Sigma_{n})}
    \]
  \end{block}

  \pause

  \begin{block}{}
    $\Sigma_{n}$ is the spinor representation space of $\Spin(n)$
  \end{block}
\end{frame}

\begin{frame}
  \frametitle{Coset spaces of $SU(2)$ - Dirac operator}
  \begin{block}{Theorem (B\"ar)}
    In general, there is an expression for the action of the Dirac operator on the representation-theoretic space.
  \end{block}

  \begin{block}{}
    The representation theory of $SU(2)$ is well-understood. Therefore once one has an expression of the Dirac operator in terms of the representation theory, obtaining the Dirac spectrum is simply a matter of calculation.
  \end{block}
\end{frame}

\begin{frame}
  \frametitle{Coset spaces of $SU(2)$ Summary of results}
  \begin{block}{Theorem (Teh)}
    The Dirac spectra of $SU(2) / \Gamma$ may be decomposed into several arithmetic progressions such that on each arithmetic progression the multiplicities of the Dirac operator are simple polynomials of the eigenvalues.
  \end{block}

  As a result, by using the Poisson summation formula, one obtains a nonperturbative expression for the spectral action:

  \[
  \Tr f(D/\Lambda) = \frac{1}{|\Gamma|}(\Lambda^3 \widehat{f}^{(2)}(0) -\frac{1}{4} \Lambda \widehat{f}(0)) + O(\Lambda^{-k})
  \]
\end{frame}

\begin{frame}
  \frametitle{Lens Spaces}
  \begin{block}{}
    Originally, I computed the spectral action for lens spaces ($\Gamma$ is a finite cyclic group) using an expression for the lens space spectrum found in the literature. Using this result, I tried to calculate the spectral action and discovered that the Dirac spectrum did not decompose into arithmetic progressions where the multiplicities became simple polynomials.
  \end{block}

  \pause

  \begin{block}{}
    The break in the pattern for the spectral action led me to the discovery that the calculation of the lens space spectrum was wrong and I subsequently corrected it.
  \end{block}
\end{frame}

\section{Coset spaces of $SU(2)$ with matter}

\begin{frame}
  \begin{block}{}
    $M$ -- Purely gravitational physical model
  \end{block}

  \begin{block}{}
    $M \times F$ -- Physical models with matter. The cardinality of $F$ indicates the number of fermions in the model.
  \end{block}
\end{frame}

\begin{frame}
  \frametitle{Spectrum}
  \begin{block}{Cisneros-Molina}
    Cisneros-Molina calculated the spectrum of the Dirac operator for the spaces $SU(2)/ \Gamma \times F$ in representation theoretic terms. Using this form of the spectrum, obtaining numerical expressions for the spectrum is simply a matter of calculation.
  \end{block}
\end{frame}

\begin{frame}
  \frametitle{Results}
  \begin{block}{Theorem (Teh)}
    The Dirac spectrum of $SU(2)/\Gamma \times F$ can be decomposed into several arithmetic progressions indexed by $\Z$ in such a way that the multiplicites of the eigenvalues form simple polynomials over each several arithmetic progression. As a result, the spectral action may be computed using the Poisson summation formula. The Poisson summation formula is given by:
    \[
      \Tr f(D / \Lambda) = \frac{N}{|\Gamma|}(\Lambda ^3 \widehat{f}^{(2)}(0) - \frac{1}{4}\Lambda \widehat{f}(0)) + O(\Lambda ^{-k}),
    \]
    where $N$ is the cardinality of the finite noncommutative space $F$.
  \end{block}
\end{frame}

\section{Bieberbach manifolds}

\begin{frame}
  \frametitle{Calculation}

  \begin{block}{Strategy}
  Use linear transformations of the parameters of the spectrum to almost cover $\Z ^3$.  By almost cover, I mean that 2-dimensional, 1-dimensional or 0-dimensional sublattices of $\Z^2$ through the origin may be covered multiple times, or not at all.
  \end{block}
\end{frame}

\begin{frame}
  \frametitle{Easiest Case: Torus}
  \begin{block}{Parameters}
    $\{(m,n,p) \in \Z ^3 \}$
  \end{block}
  \begin{block}{Eigenvalues}
    $\pm 2 \pi \sqrt{(m-a)^2 + (n-b)^2 + (p-c)^2}$
    $a$,$b$,$c$ are constants depending on the choice of spin structure.
  \end{block}
\end{frame}

\begin{frame}
  \frametitle{Summary of results}
  \begin{block}{Theorem (Teh)} All of the Bieberbach manifolds with the exception of $G5$ have Dirac spectra whose parameters almost cover $\Z ^3$. As a result the Poisson summation formula may be used to calculate the spectral action.
  \end{block}
\end{frame}

\begin{frame}
  \frametitle{Spectral Action Calculation}
  \begin{block}{}
    \begin{align*}
      \Tr f(D^2/\Lambda ^2) &= \sum_{\Z ^3} f(4 \pi ^2 ((m-a)^2 + (n-b)^2 + (p-c)^2)/\Lambda ^2) \\
      &= \sum_{\Z^3} \widehat{f}(4 \pi ^2 ((m-a)^2 + (n-b)^2 + (p-c)^2)/\Lambda ^2) \\
      &= \frac{\Lambda^3}{4\pi ^3} \int_{\R ^3}f(u^2 + v^2 + w^2)dudvdw + O(\Lambda^{-k})
    \end{align*}
  \end{block}
\end{frame}

\begin{frame}
  \frametitle{G2(a)}
    \begin{tabular}[]{lr}
\includegraphics[scale=0.15]{g2alp} & \includegraphics[scale=0.15]{g2alk}\\
    \end{tabular}

    \begin{block}{Parameters}
      $I_1 = \{(k,l,p)| k,l,p \in \Z, p > -l \}$
      $I_2 = \{(k,l,p)| k,l \in \Z l \geq 1, p = -l \}$
    \end{block}
    \begin{block}{Eigenvalues}
      $\lambda_{klm}^{\pm} = \pm 2 \pi \sqrt{\frac{(k + 1/2)^2}{H^2} + \frac{l^2}{L^2} + \frac{p^2}{S^2}}$
    \end{block}
\end{frame}

\begin{frame}
  \frametitle{G2(b)(d)(c)}
    \begin{tabular}[]{lr}
\includegraphics[scale=0.15]{g2bd} & \includegraphics[scale=0.15]{g2c}\\
    \end{tabular}
    \begin{block}{G2(b),G2(d)}
      Parameters: $\{(k,l,p)| k,l,p \in \Z, l \geq 0\}$

      Eigenvalues: $\pm 2\pi\sqrt{\frac{(k+1/2)^2}{H^2}\frac{(l+1/2)^2}{L^2} + \frac{(p+c+1/2)^2}{S^2}}$
    \end{block}
    \begin{block}{G2(c)}
      Parameters: $\{(k,l,m)| k,l,m \in \Z, m \geq 0\}$

      Eigenvalues: $\pm 2\pi\sqrt{\frac{(k+1/2)^2}{H^2}\frac{l^2}{L^2} + \frac{(m-1/2)^2}{S^2}}$
    \end{block}
\end{frame}

\begin{frame}
  \frametitle{G3, G6}
    \begin{tabular}[]{lr}
\includegraphics[scale=0.15]{g3} & \includegraphics[scale=0.15]{g6}\\
    \end{tabular}
    \begin{block}{G3}
      Parameters: $\{(k,l,m)| k,l,m \in \Z, l \geq 1, 0 \leq m < l\}$

      Eigenvalues: $\pm 2\pi\sqrt{\frac{(k+c)^2}{H^2}\frac{l^2}{L^2} + \frac{(l - 2m)^2}{L^2}}$
    \end{block}
    \begin{block}{G6}
      Parameters: $\{(k,l,m)| k,l,m \in \Z, l \geq 0, k\geq 0\}$

      Eigenvalues: $\pm 2\pi\sqrt{\frac{(k+1/2)^2}{H^2}\frac{(l+1/2)^2}{L^2} + \frac{(m + 1/2)^2}{S^2}}$
    \end{block}
\end{frame}

\begin{frame}
  \frametitle{G4(a),(b)}
    \begin{tabular}[]{lr}
\includegraphics[scale=0.15]{g4a} & \includegraphics[scale=0.15]{g4b}\\
    \end{tabular}
    \begin{block}{G4(a)}
      Parameters: $\{(k,l,p)| k,l,p \in \Z, l \geq 1, -l \leq p < l\}$

      Eigenvalues: $\pm 2\pi\sqrt{\frac{(k+1/2)^2}{H^2} + \frac{l^2 + p^2}{L^2}}$
    \end{block}
    \begin{block}{G4(b)}
      Parameters: $\{(k,l,p)| k,l,p \in \Z, l \geq 1, -l + 1 < p \leq l\}$

      Eigenvalues: $\pm 2\pi\sqrt{\frac{(k+1/2)^2}{H^2} + \frac{(l-1/2)^2 + (p - 1/2)^2}{L^2}}$
    \end{block}
\end{frame}

\begin{frame}
  \frametitle{G5}
  \begin{block}{Parameters}
    $\{(k,l,m)| k,l,m \in \Z, l \geq 1, 0 \leq m < l\}$
  \end{block}

  \begin{block}{Eigenvalues}
    $\pm 2\pi\sqrt{\frac{(k+1/2)^2}{H^2}\frac{l^2}{L^2} + \frac{(2l- m)^2}{3L^2}}$
  \end{block}

  \pause

  \begin{block}{}
    Surprisingly, the technique doesn't work for $G5$
  \end{block}
\end{frame}

\section{One-parameter family of Dirac operators on $SU(2)$ and $SU(3)$}

\begin{frame}
  \frametitle{Robertson-Walker metric on $SU(2)$}
  \begin{block}{}
    Chamseddine and Connes observed that one may apply the Euler-Maclaurin formula to compute the spectral action for $SU(2)$ equipped with the Robertson-Walker metric.
  \end{block}
\end{frame}

\begin{frame}
  \frametitle{Euler-Maclaurin formula}
  \begin{block}{}
    For a smooth function $g$,
    \begin{align*}
      \sum_{k=a}^b g(k) &= \int_a^b g(x) dx + \frac{g(a) + g(b)}{2} +  \\
      & \sum_{j=2}^m \frac{B_j}{j!}(g^{(j-1)}(b) - g^{(j-1)}(a)) - R_m
    \end{align*}
  \end{block}

  \pause

  \begin{block}{}
    \begin{itemize}
      \item Useful when Dirac spectrum is indexed by $\N$ instead of $\Z$
      \item $B_j$ are the Bernoulli numbers
      \item \[
        R_m = \frac{(-1)^m}{m!} \int_a^b g^{(m)}(x) B_m (x-[x])dx
      \]
    \end{itemize}
  \end{block}
\end{frame}

\begin{frame}
  \frametitle{Bernoulli numbers}
  Bernoulli numbers are value at 0 of Bernoulli polynomials. The Bernoulli polynomials are defined inductively as
  \[
    B_0(x) = 1, B_n'(x) = nB_{n-1}(x), \int_0^1 B_n(x)dx = 0
  \]
\end{frame}

\begin{frame}
  \frametitle{One-parameter family of Dirac operators}
  \begin{block}{One-parameter family of connections}
    \[
      \nabla_X^t = \nabla_X^0 + t[X, \cdot]
    \]
  \end{block}

  \pause

  \begin{block}{}
    \begin{itemize}
      \item Connections are metric
      \item Torsion equals $T(X,Y) = (2t-1)[X,Y]$
      \item Therefore, $t = 1/2$ corresponds to Levi-Civita connection
      \item $t = 1/3$ corresponds to cubic Dirac operator studied by Kostant
    \end{itemize}
  \end{block}
\end{frame}

\begin{frame}
  \begin{block}{Theorem (Lai, Teh)}
    The one-parameter family of Dirac operators $D_t$ may be written in terms of the Casimir operator.
    \[
      1 \otimes \Cas + (3t-1)(\Delta \Cas - 1 \otimes \Cas - \Cas \otimes 1) + 9 t^2 |\rho|^2
    \]
  \end{block}

  \begin{block}{}
    The action of $\Cas$ on irreducible representations is well-understood, and so computing the Dirac spectrum becomes once again, a matter of calculation.
  \end{block}
\end{frame}

\begin{frame}
  \begin{block}{Theorem(Teh)}
	The spectrum of the one-parameter family of Dirac operators on $SU(2)$ may be indexed by $\Z$ such that its multiplicities are 
	\begin{align*}
		& \Tr f( \frac{D_t ^2}{\Lambda ^2}) = \\ 
		&\Lambda ^3 \int_{\R}y^2 f(y^2) dy + \Lambda(3t-1)(3t-2)\int_\R f(y^2) dy + O(\Lambda^{-k})
	\end{align*}
  \end{block}
\end{frame}

\begin{frame}
  \begin{block}{Theorem(Teh)}
    The Dirac operator $D_{1/3}$ on $SU(3)$ has eigenvalues $p^2 + q^2 + pq$ for $(p,q) \in \N^2$ with multiplicities given by $2p^2q^2(p+q)^2$.
  \end{block}

  \begin{block}{}
    \begin{itemize}
      \item The strategy of transforming parameter space to cover $\Z^2$ used for Bieberbach manifolds works here as well.
      \item In this case the linear transformations of parameter space must preserve the multiplicities in addition to the eigenvalues.
    \end{itemize}
  \end{block}
\end{frame}

\begin{frame}
  \begin{block}{Theorem(Teh)}
    For $t = 1/3$, one may apply the Poisson summation formula to compute the spectrum for $SU(3)$, and it is given by:
    \begin{align*}
      & \Tr f(D_{1/3}^2)/\Lambda ^2) = \\
                                   & \frac{1}{3}\iint_{\R ^2} x^2y^2(x+y)^2f(x^2+y^2+xy)dxdy \Lambda^8 + O(\Lambda ^{-k})
    \end{align*}
  \end{block}
\end{frame}

\begin{frame}
  \begin{block}{Two variable Euler-Maclaurin formula}
    \begin{align*}
      &\sum_{p=0}^{\infty} \sum_{q=0}^{\infty} ` g(p,q) =\\
      &L^{2k} \frac{\partial}{\partial h_1}L^{2k} \frac{\partial}{\partial h_2} \int_{h1}^{\infty} \int_{h2}^{\infty} g(p,q)dpdq|_{h1=0, h2 = 0} + R_m(g) \\
    \end{align*}

    Apostrophe indicates weights of 1/2 along boundary edges, 1/4 at corner.

    \[
      L^{2k}(S) = 1 + \frac{1}{2!}B_2S^2+ \ldots + \frac{1}{(2k)!} B_{2k}S^{2k}
    \]

    {\tiny
    \begin{align*}
      &R_m(g) = \\
      & \sum_{I \subsetneq \{1,2\}}(-1)^{(m-1)(2-|I|)}\prod_{i\in I} L^{2k} \frac{\partial}{\partial h_i} \int_{h1}^{\infty}\int_{h1}^{\infty} \prod_{i \notin I} P_m(x_i) \frac{\partial}{\partial x_i} ^m g(x_1,x_2)dx_1 dx_2 |_{h=0}
    \end{align*}
    }
  \end{block}
\end{frame}

\begin{frame}
	\begin{block}{Theorem (Teh)}
		{\tiny
		\begin{align*}
	 		&\Tr f(D_t^2/ \Lambda^2) = \\
			& 2 \iint_{\R ^{+2}} f (x^2 + y^2 + xy)x^2y^2(x+y)^2 dxdy \Lambda^8 +\\
			&3(3t-1)(3t-2)\iint_{\R ^{+2}}f(x^2+y^2+xy)(x^4+2x^3y+3x^2y^2+2xy^3 +y^4)dxdy\Lambda^6 +\\
			&9(3t-1)^2(3t-2)^2\iint_{\R ^{+2}}f(x^2 + y^2+xy)(x^2+xy+y^2)dxdy\Lambda^4 +\\
			&6(3t-1)^3(3t-2)^3 \iint_{\R ^{+2}} f(x^2 + y^2 +xy)dxdy \Lambda^2+ \\
			&O(\Lambda^{-1}
		\end{align*}
		}
	\end{block}
	
	\begin{block}{}
		
	\end{block}
\end{frame}
\end{document}


\documentclass{beamer}
\usetheme{Malmoe}

\usepackage{ esint }

\def\bA{{\mathbb A}}
\def\bB{{\mathbb B}}
\def\bC{{\mathbb C}}
\def\bD{{\mathbb D}}
\def\bE{{\mathbb E}}
\def\bF{{\mathbb F}}
\def\bG{{\mathbb G}}
\def\bH{{\mathbb H}}
\def\bI{{\mathbb I}}
\def\bJ{{\mathbb J}}
\def\bK{{\mathbb K}}
\def\bL{{\mathbb L}}
\def\bM{{\mathbb M}}
\def\bN{{\mathbb N}}
\def\bO{{\mathbb O}}
\def\bP{{\mathbb P}}
\def\bQ{{\mathbb Q}}
\def\bR{{\mathbb R}}
\def\bS{{\mathbb S}}
\def\bT{{\mathbb T}}
\def\bU{{\mathbb U}}
\def\bV{{\mathbb V}}
\def\bW{{\mathbb W}}
\def\bX{{\mathbb X}}
\def\bY{{\mathbb Y}}
\def\bZ{{\mathbb Z}}

\def\A{{\mathbb A}}
\def\C{{\mathbb C}}
\def\F{{\mathbb F}}
\renewcommand{\H}{{\mathbb H}}
\def\N{{\mathbb N}}
\renewcommand{\P}{{\mathbb P}}
\def\Q{{\mathbb Q}}
\def\Z{{\mathbb Z}}
\def\R{{\mathbb R}}
\def\K{{\mathbb K}}

\def\cA{{\mathcal A}}
\def\cB{{\mathcal B}}
\def\cC{{\mathcal C}}
\def\cD{{\mathcal D}}
\def\cE{{\mathcal E}}
\def\cF{{\mathcal F}}
\def\cG{{\mathcal G}}
\def\cH{{\mathcal H}}
\def\cI{{\mathcal I}}
\def\cJ{{\mathcal J}}
\def\cK{{\mathcal K}}
\def\cL{{\mathcal L}}
\def\cM{{\mathcal M}}
\def\cN{{\mathcal N}}
\def\cO{{\mathcal O}}
\def\cP{{\mathcal P}}
\def\cQ{{\mathcal Q}}
\def\cR{{\mathcal R}}
\def\cS{{\mathcal S}}
\def\cT{{\mathcal T}}
\def\cU{{\mathcal U}}
\def\cV{{\mathcal V}}
\def\cW{{\mathcal W}}
\def\cX{{\mathcal X}}
\def\cY{{\mathcal Y}}
\def\cZ{{\mathcal Z}}

\def\Aut{{\rm Aut}}
\def\Cas{{\rm Cas}}
\def\Coker{{\rm Coker}}
\def\Diff{{\rm Diff}}
\def\dist{{\rm dist}}
\def\Dom{{\rm Dom}}
\def\Difg{{\rm Difg}}
\def\End{{\rm End}}
\def\Ext{{\rm Ext}}
\def\Gal{{\rm Gal}}
\def\GL{{\rm GL}}
\def\Gr{{\rm Gr}}
\def\Hom{{\rm Hom}}
\def\id{{\rm id}}
\def\Ind{{\rm Ind}}
\def\Index{{\rm Ind}}
\def\Inf{{\rm Inf}}
\def\Inn{{\rm Inn}}
\def\Int{{\rm Int}}
\def\Ker{{\rm Ker}}
\def\Lie{{\rm Lie}}
\def\Li{{\rm Li}}
\def\Lim{{\rm Lim}}
\def\Mod{{\rm Mod}}
\def\Out{{\rm Out}}
\def\PGL{{\rm PGL}}
\def\PSL{{\rm PSL}}
\def\rank{{\rm rank}}
\def\Res{{\rm Res}}
\def\Rep{{\rm Rep}}
\def\sign{{\rm sign}}
\def\SL{{\rm SL}}
\def\Spec{{\rm Spec}}
\def\Sp{{\rm Spec}}
\def\spin{{\rm spin}}
\def\Spin{{\rm Spin}}
\def\SU{{\rm SU}}
\def\Sup{{\rm Sup}}
\def\Trace{{\rm Tr}}
\def\Tr{{\rm Tr}}
\def\tr{{\rm tr}}
\def\Vect{{\rm Vect}}

\def\elel{(  \downarrow 1)}
\def\dodo{(   \downarrow 3)}
\def\nunu{(   \uparrow 1)}
\def\upup{(   \uparrow 3)}
\def\mass{Y}

\def\fa{{\mathfrak a}}
\def\fb{{\mathfrak b}}
\def\fc{{\mathfrak c}}
\def\fd{{\mathfrak d}}
\def\fe{{\mathfrak e}}



\title{Dirac Spectra, Summation Formulae, and the Spectral Action} \author{Kevin Teh} \date{May 17, 2013}



\title{Dirac Spectra, Summation Formulae, and the Spectral Action}
\author{Kevin Teh}
\institute{Caltech}
\date{May 17, 2013}

\begin{document}

\begin{frame}
\titlepage
\end{frame}

\section{Intro to NCG and spectral action}

\subsection{Spectral Triples}
\begin{frame}
  \frametitle{Spectral Triples - Definition}
  \begin{block}{Definition}
    A spectral triple $(\cA, \cH, D)$ is given by an involutive unital algebra $\cA$ represented as operators in a Hilbert space $\cH$ and a self-adjoint operators $D$ with compact resolvent such that all commutators $[D, a]$ are bounded for $a \in \cA$.
  \end{block}
  \pause

  \begin{block}{}
    A spectral triple is even if the Hilbert space $\cH$ is equipped with a $\Z /2 $-grading $\gamma$ which commutes with any $a \in \cA$ and anti-commutes with $D$.
\end{block}
\end{frame}


\begin{frame}
  \frametitle{Spectral Triples - Example}
  \begin{exampleblock}{Compact spin manifold $M$}
    \begin{itemize}
      \item $\cA = C^{\infty}(M)$
      \item $\cH = L^2(M, S)$, square-integrable sections of spinor bundle
      \item $D$ Dirac operator associated with spin structure
    \end{itemize}
  \end{exampleblock}
\end{frame}

\subsection{Spectral Action}

\begin{frame}
  \frametitle{Spectral Action}

  \begin{block}{}
    For a spectral triple, $(\cA, \cH, D)$, the spectral action is given by
    \[
    \Tr f(D / \Lambda)
    \]
  \end{block}

  \pause

  \begin{itemize}
    \item Typically, $f$ is a smooth approximation to a cutoff function.
    \item For our purposes, $f$ is an even function, $f \in \mathcal{S} (\R)$
    \item $\Lambda > 0$
    \item It is equivalent to calculate $f(D^2/ \Lambda ^2)$, since we may regard this simply as a different choice of the function $f$.
  \end{itemize}

  \pause

  \begin{block}{}
    The spectral action plays a central role in the Noncommutative Standard Model
  \end{block}

\end{frame}


\subsection{Noncommutative Standard Model}

\begin{frame}
  \frametitle{Noncommutative Standard Model}

  \begin{block}{}
    A principal application of noncommutative geometry is a formulation of the standard model of physics using the mathematical framework of noncommutative geometry.
  \end{block}

  \pause

  \begin{block}{}
    Simple input for the model: the finite-dimensional algebra $\C \oplus \H \oplus \H \oplus M_3(\C)$
  \end{block}

  \pause

  \begin{block}{}
    Spacetime is modeled as the product geometry of a finite-dimensional spectral triple of the above algebra with a 4-dimensional compact Riemannian spin manifold and one may recover the usual standard model
  \end{block}

\end{frame}


\begin{frame}
  \frametitle{Noncommutative Standard Model and Spectral Action}
  \begin{block}{}
    By computing the spectral action for ``inner fluctuations'' of the Dirac operator, one obtains the gauge bosons of the theory.
  \end{block}

  \begin{block}{}
    To give a flavour of what this involves, one must first equip the geometry with a real structure, $J$ (an antilinear unitary operator). Inner fluctuations of the Dirac operator are given by
    \[
    D_A = D + A + J A J^{-1},~ A = \sum a_j [D, b_j],~ a_j, b_j \in \cA,~ A = A^*
    \]
  \end{block}
\end{frame}

\subsection{Noncommutative Cosmology}

\section{Spectral Action}

\subsection{Perturbative vs Non-perturbative}

\begin{frame}
  \frametitle{Perturbative vs Non-perturbative}
  \begin{block}{Perturbative expansion of spectral action via Gilkey's Theorem}
    \[
      \Tr f (D/\Lambda) \sim 2 \Lambda^4 f_4 a_0 + 2 \Lambda^2 f_2 a_2 + f_0 a_4 + \cdots + \Lambda^{-2k} f_{-2k} a_{4+2k} + \cdots
    \]

    \begin{align*}
      f_k &= \int_{0}^{\infty} f(u) u^{k-1} du, ~ k>0 \\
      &= f(0), ~ k = 0\\
      &= (-1)^{-k/2} \frac{(-k/2)!}{-k!}f^{(-k)}(0), ~ k < 0
    \end{align*}
  \end{block}
\end{frame}

\begin{frame}
	\frametitle{Aside on Poisson Summation Formula}
	\begin{block}{Poisson Summation Formula}
		\[
			\sum_{n \in \Z} h(n) = \sum_{n\in \Z} \widehat{h}(n),
		\]
		$\widehat{h}(x) = \int_{\R} h(u) e^{2\pi i u x} du$
	\end{block}
	
	\pause
	
	\begin{block}{Slightly more general form}
		\[
		\sum_{n \in \Z} h(x + \lambda n) = \frac{1}{\lambda} \sum_{n \in \Z} e^{\frac{2\pi i n x}{\lambda}}\widehat{h}(\frac{n}{\lambda})
		\]
	\end{block}
	
	
\end{frame}

\begin{frame}
  \frametitle{Perturbative vs Non-perturbative}
  \begin{block}{Nonperturbative calculation of the spectral action}
    \[
      \Tr f (D/\Lambda) = \sum_{n \in \Z} h(x + \lambda n) = \frac{1}{\lambda} \sum_{n \in \Z} e^{\frac{2\pi i n x}{\lambda}}\widehat{h}(\frac{n}{\lambda})
    \]
  \end{block}

  \pause

  \begin{block}{}
	In this form, the spectral action provides a slow-roll inflation potential.
  \end{block}
\end{frame}

\subsection{Slow-roll Potential}
\begin{frame}
	\begin{block}{}
		Substitute $D^2 \mapsto D^2 + \phi^2$. On $S^3 \times S^1$ this leads to a shift in the spectral action of
		\[
		-\pi \Lambda ^2 \beta a^3 \int_0^{\infty} h(v)dv \phi ^2 + \frac{1}{2} \pi \beta a h(0) \phi^2 + \frac{1}{2} \pi \beta a^3 h(0) \phi^4.
		\]
	\end{block}
	
	\begin{block}{}
		This quantity is physically interpreted as a slow-roll inflation potential.  The form of the potential and its corresponding slow-roll parameters are sensitive to the cosmic geometry.
	\end{block}
\end{frame}

\begin{frame}
  \frametitle{The First Example: $SU(2)$}
  \begin{block}{Usual Expression of Dirac Spectrum}
    $\rm{Spec}(D) = \{ \pm (\frac{3}{2} + k) | k \in \Z, k \geq 0\}$

    multiplicity of $(\frac{3}{2} + k)$ is $2 {k+2} \choose {k}$
  \end{block}

  \pause

  \begin{block}{Alternative Expression}
    $\rm{Spec}(D) = \{n + 1/2 | n \in \Z \}$

    multiplicity of $n + 1/2$ is $n(n+1)$
  \end{block}

  \pause

  \begin{block}{}
    Important features of alternative expression:
    \begin{itemize}
      \item spectrum is indexed by $\Z$
      \item multiplicity is a polynomial, and in particular, smooth
    \end{itemize}
  \end{block}
\end{frame}

\begin{frame}
  \frametitle{The First Example: $SU(2)$ aka $S^3$}
  \begin{block}{}
    Define $g(u) = (u^2 -1/4)f(u/\Lambda)$
  \end{block}

  \pause

  \begin{block}{}
    \begin{align*}
      \widehat{g}(x) &= \int_{\R}g(u) e^{-2 \pi i x u}du = \int_{\R}(u^2 - \frac{1}{4})f(u/\Lambda)e^{-2 \pi i x u}du \\
      &= \Lambda ^3 \int_{\R}v^2 f(v) e^{-2\pi i \Lambda x v}dv - \frac{1}{4} \Lambda \int_{\R}f(v)e^{-2\pi i \Lambda xv}dv
    \end{align*}
  \end{block}

  \pause

  \begin{block}{}
    Applying the Poisson summation formula to $g$ yields

    \[
      Tr(f(D/\Lambda)) = \Lambda^3 \sum_{\Z}(-1)^n \widehat{f}^{(2)}(\Lambda n) - \frac{1}{4} \Lambda \sum_{\Z}(-1)^n \widehat{f}(\Lambda n)
    \]
  \end{block}
\end{frame}

\begin{frame}
  \frametitle{The First Example: $SU(2)$ aka $S^3$}
  \begin{block}{}
    Since we took $f \in \cS (\R)$, we have the estimates

    \[
      \sum_{n \neq 0}|\widehat{f}(\Lambda n)| \leq C_k \Lambda^{-k}, ~ \sum_{n \neq 0}|\widehat{f}^{(2)}(\Lambda n)| \leq C_k \Lambda^{-k}
    \]
  \end{block}

  \pause

  \begin{block}{}
    This gives the following theorem (Chamseddine, Connes)

    \[
      Tr(f(D/\Lambda)) = \Lambda^3 \int_\R v^2 f(v) dv - \frac{1}{4}\Lambda \int_{\R}f(v)dv + O(\Lambda)^{-k}
    \]
  \end{block}

  \pause

  \begin{block}{}
    Concrete estimates of the remainder are possible
  \end{block}
\end{frame}

\begin{frame}
  \begin{block}{}
  We want to know: Are there other situations where this technique works?
  \end{block}
\end{frame}

\begin{frame}
  \frametitle{Nonexample: $S^4$}
  \begin{block}{Spectrum}
    $\Spec(D) = \{\pm (2 + k) | k \in \Z k \geq 0 \}$

    $\rm{multiplicity} = 4 {{k+3} \choose {k}}$
  \end{block}

  \pause

  \begin{alertblock}{The technique doesn't work}
    When one tries to parametrize the spectrum by $\Z$, the absolute value $|x|$ appears in the expression for the multiplicity. The resulting function is no longer smooth, and its Fourier transform no longer has the nice property of being in $\cS$
  \end{alertblock}
\end{frame}

\begin{frame}
  \frametitle{Spaces that we will consider}
  \begin{itemize}
    \item Coset spaces of $SU(2)$
    \item Coset spaces of $SU(2) \times F$
    \item Bieberbach manifolds (Quotients of flat $\mathbb{T} ^3$)
    \item $SU(3)$
  \end{itemize}

\end{frame}

\section{Coset spaces of $SU(2)$}

\begin{frame}
	\frametitle{$SU(2)/Q8$}
	\begin{block}{Theorem(Ginoux)}
	$SU(2)/Q8$ has four spin structures. For one spin structure, the Dirac operator has spectrum (with parameter $k \in \Z$)
	\[
	\left\{ \begin{array}{lll}
\frac{3}{2} + 4k & \text{with multiplicity} & 2(k+1)(2k+1) \\[2mm]
\frac{3}{2} + 4k +2 & \text{with multiplicity} & 4k(k+1),
\end{array}\right.
	\]
	
	For the other three spin structures, the spectrum is
	\[
\left\{ \begin{array}{lll}
\frac{3}{2} + 4k & \text{with multiplicity} & 2k(2k+1) \\[2mm]
\frac{3}{2} + 4k +2 & \text{with multiplicity} & 4(k+1)^2.
\end{array}\right.
	\]
	\end{block}
\end{frame}

\begin{frame}
	\frametitle{$SU(2)/Q8$}
	\begin{block}{Theorem(Teh)}
		One may find polynomials which return the multiplicities when evaluated at the eigenvalues. For the first spectrum the polynomials are
		
		\[
P_1(u) =\frac{1}{4} u^2 + \frac{3}{4} u + \frac{5}{16}, \quad P_2(u) =  \frac{1}{4} u^2 - \frac{3}{4} u - \frac{7}{16}
		\]
		
		For the second spectrum they are
		
		\[
P_1(u)  = \frac{1}{4} u^2 - \frac{1}{4} u - \frac{3}{16}, \quad P_2(u)  = \frac{1}{4} u^2 + \frac{1}{4} u + \frac{1}{16}
		\]
	\end{block}
\end{frame}

\begin{frame}
	\frametitle{$SU(2)/Q8$}
	\begin{block}{Theorem (Teh)}
		Define $g_i(u) = P_i(u)f(u/\Lambda)$. Applying the Poisson summation formula to $g_1, g_2$ results in the expression of the spectral action
		
		\[
		\Tr f(D/\Lambda) = \frac{1}{8}\Lambda^3 \widehat{f}^{(2)}(0) - \frac{1}{32}\Lambda \widehat{f}(0) + O(\Lambda^{-k})
		\]
	\end{block}
	
	\pause
	
	\begin{block}{}
		The same result holds for all four spin structures. In all cases where I have computed the spectral action for multiple spin structures (see Bieberbach manifolds), the expression for the spectral action is independent of the spin structure, even though the Dirac spectrum in general depends on the choice of spin structure.
	\end{block}
\end{frame}

\begin{frame}
	\frametitle{Poincar\'e Homology Sphere}
	\begin{block}{Generating functions}
	\begin{align*}
F_+(z) & = \sum_{k=0}^{\infty} m (\frac{n}{2}+k,D)z^k  \\
F_-(z) &= \sum_{k=0}^{\infty} m (-(\frac{n}{2}+k),D)z^k .
\end{align*}
	\end{block}
\end{frame}

\begin{frame}
	\begin{block}{Formula for the generating functions}
		\begin{align*}
F_+ (z) &= \frac{1}{| \Gamma |} \sum_{\gamma \in \Gamma}  \frac{\chi^-(\epsilon(\gamma)) - z \cdot \chi^+ (\epsilon(\gamma))}{\det(1_{2m} - z \cdot \gamma)}, \\ \label{Spmultgen2}
F_-(z) &= \frac{1}{| \Gamma |} \sum_{\gamma \in \Gamma} \frac{\chi^+(\epsilon(\gamma)) - z \cdot \chi^- (\epsilon(\gamma))}{\det(1_{2m} - z \cdot \gamma)}.
		\end{align*}
	\end{block}
	
	\begin{block}{}
	\begin{itemize}
		\item $\Gamma$ -- subgroup of $SO(n)$
		\item $\chi^{\pm}$ -- characters of half spin representations of $\Spin(4)$
		\item $\epsilon$ -- group homomorphism $SO(n) \rightarrow \Spin(4)$ depending on choice of spin structure
	\end{itemize}
	\end{block}
\end{frame}

\begin{frame}
	\frametitle{Identifying unit quaternions with elements of  $SO(4)$}
	\begin{block}{Unit Quaternion}
	$a + bi + cj + dk$, $a^2 + b^2 + c^2 + d^2 = 1$
	\end{block}
	\begin{block}{Element of $SO(4)$}
	\[
\left(
\begin{array}{cccc}
a & -b & -c & -d \\
b & a & -d & c \\
c & d & a & -b \\
d & -c & b & a \\
\end{array}
\right).
\]
	\end{block}
\end{frame}

\begin{frame}
	\frametitle{The double cover $\Spin(4) \rightarrow SO(4)$}
	\begin{block}{$\Spin(4) \simeq S^3_L \times S^3_R$}
	\end{block}
	\begin{block}{$S^3_L, S^3_R$}
$\left(
\begin{array}{cccc}
a & -b & -c & -d \\
b & a & -d & c \\
c & d & a & -b \\
d & -c & b & a \\
\end{array}
\right)$, $\left(
\begin{array}{cccc}
p & -q & -r & -s \\
q & p & s & -r \\
r & -s & p & q \\
s & r & -q & p
\end{array}
\right)$	
	\end{block}
	\begin{block}{Double cover}
	$(A,B) \mapsto AB$
	\end{block}
\end{frame}

\begin{frame}
\frametitle{Half-spin representations}
\begin{block}{$\rho^+$}
\[
\left(
\begin{array}{cccc}
a & -b & -c & -d \\
b & a & -d & c \\
c & d & a & -b \\
d & -c & b & a \\
\end{array}
\right)
\mapsto
\left(
\begin{array}{cc}
a -bi & d + ci \\
-d + ci & a + bi
\end{array}
\right)
\]
\end{block}

\begin{block}{$\rho^-$}
\[
\left(
\begin{array}{cccc}
p & -q & -r & -s \\
q & p & s & -r \\
r & -s & p & q \\
s & r & -q & p
\end{array}
\right)^{t}
\mapsto
\left(
\begin{array}{cc}
p - qi & s + ri \\
-s + ri & p + qi
\end{array}
\right)
\]
\end{block}
\end{frame}

\begin{frame}
\frametitle{Evaluating the generating function}
	\begin{block}{Formula for the generating functions}
		\begin{align*}
F_+ (z) &= \frac{1}{| \Gamma |} \sum_{\gamma \in \Gamma}  \frac{\chi^-(\epsilon(\gamma)) - z \cdot \chi^+ (\epsilon(\gamma))}{\det(1_{2m} - z \cdot \gamma)}, \\ \label{Spmultgen2}
F_-(z) &= \frac{1}{| \Gamma |} \sum_{\gamma \in \Gamma} \frac{\chi^+(\epsilon(\gamma)) - z \cdot \chi^- (\epsilon(\gamma))}{\det(1_{2m} - z \cdot \gamma)}.
		\end{align*}

		\begin{align*}
			\chi^- (\epsilon(\gamma)) &= 2a \\
			\chi^+ (\epsilon(\gamma)) &= 2.
		\end{align*}
	\end{block}
\end{frame}

\begin{frame}
	\frametitle{Generating functions for Poincar\'e Homology Sphere}
	\begin{block}{$F_+$}
	\[
F_+(z) = -\frac{16(710647 + 317811 \sqrt{5})G^+(z)}{(7 + 3 \sqrt{5})^3 (2207 + 987 \sqrt{5})H^+(z)},
\]
{\small
$$ \begin{array}{rl}
G^+(z) =&  6z^{11} + 18z^{13} + 24z^{15} + 12z^{17} - 2z^{19} \\[2mm]
- & 6z^{21} - 2z^{23} + 2 z^{25} + 4z^{27} + 3z^{29} + z^{31},
\end{array} $$
$$ \begin{array}{rl}
H^+(z) = & -1 -3 z^{2}  -4z^{4}-2z^{6}+2z^{8}+ 6z^{10} + 9z^{12} + 9z^{14} + 4 z^{16}\\[2mm]
- &  4 z^{18} - 9 z^{20} -9z^{22}-6z^{24}-2z^{26} + 2z^{28} + 4z^{30} + 3z^{32} + z^{34},
\end{array}
$$}
	\end{block}
\end{frame}

\begin{frame}
	\frametitle{Generating functions for Poincar\'e Homology Sphere}
	\begin{block}{$F_-$}
\[
F_-(z) = -\frac{1024(5374978561 + 2403763488  \sqrt{5})G^-(z)}{(7 + 3 \sqrt{5})^8 (2207 + 987 \sqrt{5})H^-(z)},
\]
{\small
$$ \begin{array}{rl}
G^-(z) = & 1 + 3z^{2} + 4z^{4} + 2z^{6} - 2z^{8}-6z^{10} \\[2mm]
- & 2z^{12} + 12 z^{14} + 24z^{16} + 18z^{18} + 6z^{20},
\end{array}
$$
$$ \begin{array}{rl}
H^-(z) = & -1 -3 z^{2}  -4z^{4}-2z^{6}+2z^{8}+ 6z^{10} + 9z^{12} + 9z^{14} + 4 z^{16}
\\[2mm] - & 4 z^{18}  - 9 z^{20} - 9z^{22}-6z^{24}-2z^{26} + 2z^{28} + 4z^{30} + 3z^{32} + z^{34}. \end{array} $$}
	\end{block}
\end{frame}

\begin{frame}
\frametitle{Poincar\'e Homology Sphere - Spectral Action}
\begin{block}{Theorem(Teh)}
The spectrum of the Poincar\'e homology sphere may be decomposed into 60 arithmetic progressions such that the multiplicities of the eigenvalues are quadratic polynomials. The sum of these polynomials is
\[
\sum_{j=0}^{59} P_j(u) =  \frac{1}{2}u^2-\frac{1}{8}. 
\]
Therefore, when we apply the Poisson summation formula, the spectral action becomes
\[
\Tr(f(D/\Lambda)) = \frac{1}{60} \left(  \frac{1}{2}\Lambda^3 \widehat f^{(2)}(0) -\frac{1}{8}\Lambda \widehat f(0)  \right),
\]
\end{block}
\end{frame}

\begin{frame}
  \frametitle{Coset spaces of $SU(2)$}
  \begin{block}{}
    The spaces we are considering are the Riemannian homogeneous spaces $SU(2)/\Gamma$ where $\Gamma$ is a finite subgroup of $SU(2)$.

    Each of the following groups are subgroups of $SU(2)$ and are unique up to conjugacy:
    \begin{itemize}
      \item cyclic group of order $N$, $N = 1,2,3,\ldots$
      \item dicyclic group of order $4N$, $N = 2,3,\ldots$
      \item binary tetrahedral group
      \item binary octahedral group
      \item binary icosahedral group
    \end{itemize}
  \end{block}
\end{frame}

\begin{frame}
  \frametitle{Coset spaces of $SU(2)$ - Spectrum}
  \begin{block}{Theorem(B\"ar)}
    The Hilbert space of $L^2$-spinors of $G/H$ may be given the representation theoretic form

    \[
      \overline{\oplus_{\gamma \in \hat{G}} V_{\gamma} \otimes \Hom_{H}(V_{\gamma}, \Sigma_{n})}
    \]
    
    The Dirac operator acts on each summand as $\id \otimes D_{\gamma}$, where
    \[
    D_{\gamma}(A) := - \sum_{k=1}^n e_k \cdot A \circ (\pi_{\gamma})_* (X_k)
+ \left( \sum_{i=1}^n \beta_i e_i + \sum_{i<j<k}\alpha_{ijk}e_i \cdot e_j \cdot e_k \right) \cdot A.
    \]
  \end{block}
\end{frame}

\begin{frame}
  \begin{block}{Theorem(B\"ar)}
  When $G = SU(2)$, $H$ is a finite subgroup of $SU(2)$, and $G/H$ is equipped with the Berger metric corresponding to the parameter $T>0$, this becomes
  \[
D_n A = - \sum _k e_k \cdot A \cdot (\pi_n)_*(X_k) - \left( \frac{T}{2} + \frac{1}{T} \right).
\]
  \end{block}
\end{frame}

\begin{frame}
	\frametitle{Representation theory of $SU(2)$}
  \begin{block}{Irreducible representations}
	$V_n$, space of homogenous complex polynomials of degree $n$ in two variables $z_1$, $z_2$.
	
	Basis: $P_k(z_1,z_2) = z_1^{n-k}z_2^k$
  \end{block}
  
  \begin{block}{Action of $SU(2)$}
  	For $A \in SU(2)$,
	\[
	(AP)(z) = P(A\cdot z) 
	\]
  \end{block}
\end{frame}

\begin{frame}
	\frametitle{Spinor space $\Sigma_3$}
	\begin{block}{$\Sigma_3 \simeq \C ^2$}
	\end{block}
	\begin{block}{Action of $SU(2)$ on $\Sigma_3$}
		Matrix multiplication
	\end{block}
\end{frame}

\begin{frame}
	\begin{block}{Basis of $\Hom_{\C}(V_n, \Sigma_3)$}
	For $k = 0,1,\ldots n$,
	\begin{align*}
A_k(P_l) & = \begin{cases}
\left( \begin{array}{c}
1\\
0
\end{array} \right),
 & \mbox{if } k=l,~k\mbox{ is even} \\
\left( \begin{array}{c}
0\\
1
\end{array} \right),
& \mbox{if } k=l,~k\mbox{ is odd}\\
\quad 0,
& \mbox{otherwise}
 \end{cases} \\
B_k(P_l) & = \begin{cases}
\left( \begin{array}{c}
0\\
1
\end{array} \right),
 & \mbox{if } k=l,~k\mbox{ is even} \\
\left( \begin{array}{c}
1\\
0
\end{array} \right),
& \mbox{if } k=l,~k\mbox{ is odd}\\
\quad 0,
& \mbox{otherwise}
 \end{cases}
\end{align*}
	\end{block}
\end{frame}

\begin{frame}
	\begin{block}{Theorem (B\"ar)}
		On $SU(2)/ H$ equipped with the Berger metric (parameter $T>0$), $D$ acts on $V_n \otimes \Hom_{H}(V_{\gamma},\Sigma_n)$ as $\id \otimes D_n$, where $D_n = D_n' -(T/2 + 1/T)$, and
	\begin{align*}
	D_n ' A_k & = \frac{1}{T}(n-2k)A_k + 2(k+1)A_{k+1},~\rm{k~even} \\
	D_n ' A_k & = 2(n+1-k) A_{k-1} + \frac{1}{T}(2k - n)A_k,~\rm{k~odd}\\
	D_n ' B_k & = 2(n+1-k) B_{k-1} + \frac{1}{T}(2k - n)B_k,~\rm{k~even}\\
	D_n ' B_k & = \frac{1}{T}(n-2k)B_k + 2(k+1)B_{k+1},~\rm{k~odd}.
	\end{align*}	
	\end{block}
\end{frame}

\begin{frame}
\frametitle{Lens space $H = \Z_N$}
\begin{block}{}
Using his results on the Dirac operator, B\"ar goes on to compute the Dirac spectrum of lens spaces. There was a mistake. Which I corrected.
\end{block}
\pause
\begin{block}{}
To compute the spectrum, the main task is to determine $\Hom_{\Z_N}(V_n, \Sigma_3)$. I.e. which linear transformations commute with the generator of $\Z_N$,
\[
B =
\left(
\begin{array}{cc}
e^{\frac{2 \pi i}{N}} & 0 \\
0 & e^{\frac{-2 \pi i}{N}}
\end{array}
\right)
\]
\end{block}
\end{frame}

\begin{frame}
For a linear transformation $f \in \Hom_{\C}(V_n, \Sigma_3)$, let
\[
\left(
\begin{array}{c}
f_{1k} \\
f_{2k}
\end{array}
\right)
:= f(P_k).
\]
Then $f$ commutes with $B$ translates to
\[
\left(
\begin{array}{c}
f_{1k} \\
f_{2k}
\end{array}
\right)
=
\left(
\begin{array}{c}
e^{2\pi i \frac{2k-n+1}{N}}f_{1k} \\
e^{2\pi i \frac{2k-n-1}{N}}f_{2k}
\end{array}
\right).
\]
\end{frame}

\begin{frame}
It follows that $\Hom _{\mathbb{Z}_N} (V_{n}, \Sigma_3)$  has the following basis:
\begin{align*}
& \{A_k : k = \frac{mN+n-1}{2} \in \{ 0,1,\ldots, n\}, m\in \Z,k \rm{~even} \} \\
& \cup \{B_k : k = \frac{mN+n-1}{2} \in \{ 0,1,\ldots, n\}, m\in \Z, k \rm{~odd} \} \\
& \cup \{A_k : k = \frac{mN+n+1}{2} \in \{ 0,1,\ldots, n\}, m\in \Z, k \rm{~odd} \} \\
& \cup \{B_k : k = \frac{mN+n+1}{2} \in \{ 0,1,\ldots, n\}, m\in \Z, k \rm{~even} \}. \\
\end{align*}
\end{frame}

\begin{frame}
One now determines the action of $D_n'$ on this basis using the formulas above. With respect to this basis, $D_n'$ is block diagonal with blocks of size size 1 or 2. The eigenvalues of these blocks may be computed directly, and as a result we obtain the spectrum.
\end{frame}

\begin{frame}
\begin{block}{Theorem (Teh)}
	The Dirac spectrum of lens space equipped with the Berger metric $T>0$ for $N$ even is
	
	\begin{tabular}{|c|c|}
\hline
$\lambda$ & multiplicity\\
\hline
$\{ -\frac{T}{2} \pm  \sqrt{(1+n)^2 + m^2 N^2\left(\frac{1}{T^2} - 1 \right)} |$    & \\
$n \in 2\N +1, m\in \Z, -n \leq mN-1 < n \}$ & $n+1$\\
\hline
$\{ -\frac{T}{2} -\frac{mN}{T} | m \in \N \}$ & $2mN$ \\
\hline
\end{tabular}

For $N$ odd it is
\begin{tabular}{|c|c|}
\hline
$\lambda$ & multiplicity\\
\hline
$\{ -\frac{T}{2} \pm  \sqrt{(1+n)^2 + m^2 N^2\left(\frac{1}{T^2} - 1 \right)}|$ & \\ $(n \in 2 \N +1, m \in 2\Z)$ \rm{or}  & \\$(n \in 2\N, $m$ \in 2\Z + 1)$, $-n \leq mN-1 < n \}$  &$ n+1$\\
\hline
$\{ -\frac{T}{2} -\frac{mN}{T}| m \in \N$ \}  & $2mN$ \\
\hline
\end{tabular}
\end{block} 
\end{frame}

\begin{frame}
  \frametitle{Coset spaces of $SU(2)$ Summary of results}
  \begin{block}{Theorem (Teh)}
    For each subgroup $\Gamma$, the Dirac spectra of $SU(2) / \Gamma$ may be decomposed into several arithmetic progressions such that on each arithmetic progression the multiplicities of the Dirac operator are simple polynomials of the eigenvalues.
  \end{block}

  As a result, by using the Poisson summation formula, one obtains a nonperturbative expression for the spectral action:

  \[
  \Tr f(D/\Lambda) = \frac{1}{|\Gamma|}(\Lambda^3 \widehat{f}^{(2)}(0) -\frac{1}{4} \Lambda \widehat{f}(0)) + O(\Lambda^{-k})
  \]
\end{frame}

\begin{frame}
	\frametitle{Calculations of Dirac spectrum and patterns in the spectral action}
\end{frame}

\section{Coset spaces of $SU(2)$ with matter}

\begin{frame}
  \begin{block}{}
    $M$ -- Purely gravitational physical model
  \end{block}

  \begin{block}{}
    $M \times F$ -- Physical models with matter. The cardinality of $F$ indicates the number of fermions in the model.
  \end{block}
\end{frame}

\begin{frame}
\frametitle{Definitions}
\begin{block}{Exponent $c_{\Gamma}$}
$c_{\Gamma}$ is the exponent of the group $\Gamma$, the least common multiple of the orders of the elements in $\Gamma$
\end{block}
\begin{block}{$E_k$}
$E_k$ is the dimension $k+1$ irreducible representation of $SU(2)$ of homogenous complex polynomials in two variables of order $k$
\end{block}
\end{frame}

\begin{frame}
  \frametitle{$SU(2)/\Gamma \times F$}
  \begin{block}{Theorem (Cisneros-Molina)}
Let $\alpha :\Gamma \rightarrow GL_N(\C)$ be a representation of $\Gamma$.  Then the eigenvalues of the Dirac operator $D_{\alpha}^{\Gamma}$ on $S^3 / \Gamma$ are

\begin{align*}
-\frac{1}{2}-(k+1)~ &\mbox{with~multiplicity}~ \langle \chi_{E_{k+1}}, \chi_{\alpha} \rangle_{\Gamma} (k + 1), \quad &k \geq 0, \\
-\frac{1}{2}+(k+1)~ &\mbox{with~multiplicity}~ \langle \chi_{E_{k-1}}, \chi_{\alpha} \rangle_{\Gamma} (k + 1), \quad &k \geq 1.
\end{align*}
  \end{block}
\end{frame}

\begin{frame}
\begin{block}{Theorem(Cisneros-Molina)}
Let $k = c_{\Gamma} l + m$ with $0 \leq m < c_{\Gamma}$.
\begin{enumerate}
\item If $ -1 \in \Gamma$, then
\[
\langle \chi_{E_k}, \chi_{\alpha} \rangle_{\Gamma} =
\left\{
	\begin{array}{ll}
		\frac{c_{\Gamma} l}{|\Gamma |} (\chi_{\alpha}(1) + \chi_{\alpha}(-1)) + \langle \chi_{E_m}, \chi_{\alpha} \rangle_{\Gamma}  & \mbox{if $k$ is even}  \\
		\frac{c_{\Gamma} l}{|\Gamma |} (\chi_{\alpha}(1) - \chi_{\alpha}(-1)) + \langle \chi_{E_m}, \chi_{\alpha} \rangle_{\Gamma} & \mbox{if $k$ is odd}.
	\end{array}
\right.
\]
\item If $-1 \notin \Gamma$, then
\[
\langle \chi_{E_k}, \chi_{\alpha} \rangle_{\Gamma} = \frac{N c_{\Gamma}l}{\#\Gamma} + \langle \chi_{E_m}, \chi_{\alpha} \rangle_{\Gamma}.
\]
\end{enumerate}
\end{block}

\end{frame}

\begin{frame}
  \frametitle{Results}
  \begin{block}{Theorem (Teh)}
    The Dirac spectrum of $SU(2)/\Gamma \times F$ can be decomposed into several arithmetic progressions indexed by $\Z$ in such a way that the multiplicites of the eigenvalues form simple polynomials over each several arithmetic progression. As a result, the spectral action may be computed using the Poisson summation formula. The Poisson summation formula is given by:
    \[
      \Tr f(D / \Lambda) = \frac{N}{|\Gamma|}(\Lambda ^3 \widehat{f}^{(2)}(0) - \frac{1}{4}\Lambda \widehat{f}(0)) + O(\Lambda ^{-k}),
    \]
    where $N$ is the cardinality of the finite noncommutative space $F$.
  \end{block}
\end{frame}

\begin{frame}
\frametitle{Sketch of calculation}
By decomposing $\chi_{\alpha}$ as a sum of irreducible characters, we may assume that $\alpha$ is an irreducible representation. Main task is to compute the inner products
\[
\beta_m^{\alpha} = \langle \chi_{E_{m-1}}, \chi_{\alpha} \rangle_{\Gamma},
\]

\[
\gamma_m^{\alpha} = \langle \chi_{E_{m+1}}, \chi_{\alpha} \rangle_{\Gamma},
\]

\pause

\begin{block}{}
	I.e. we need to decompose $\chi_{E_k}$ into irreducible characters.
\end{block}
\end{frame}

\section{Bieberbach manifolds}

\begin{frame}
  \frametitle{Bieberbach manifolds}

  \begin{block}{Strategy}
  Use linear transformations of the parameters of the spectrum to almost cover $\Z ^3$.  By almost cover, I mean that 2-dimensional, 1-dimensional or 0-dimensional sublattices of $\Z^2$ through the origin may be covered multiple times, or not at all.
  \end{block}
\end{frame}

\begin{frame}
  \frametitle{Easiest Case: Torus}
  \begin{block}{Parameters}
    $\{(m,n,p) \in \Z ^3 \}$
  \end{block}
  \begin{block}{Eigenvalues}
    $\pm 2 \pi \sqrt{(m-a)^2 + (n-b)^2 + (p-c)^2}$
    $a$,$b$,$c$ are constants depending on the choice of spin structure.
  \end{block}
\end{frame}

\begin{frame}
  \frametitle{Summary of results}
  \begin{block}{Theorem (Teh)} All of the Bieberbach manifolds with the exception of $G5$ have Dirac spectra whose parameters almost cover $\Z ^3$. As a result the Poisson summation formula may be used to calculate the spectral action.
  \end{block}
\end{frame}

\begin{frame}
  \frametitle{Spectral Action Calculation}
  \begin{block}{}
    \begin{align*}
      \Tr f(D^2/\Lambda ^2) &= \sum_{\Z ^3} f(4 \pi ^2 ((m-a)^2 + (n-b)^2 + (p-c)^2)/\Lambda ^2) \\
      &= \sum_{\Z^3} \widehat{f}(4 \pi ^2 ((m-a)^2 + (n-b)^2 + (p-c)^2)/\Lambda ^2) \\
      &= \frac{\Lambda^3}{4\pi ^3} \int_{\R ^3}f(u^2 + v^2 + w^2)dudvdw + O(\Lambda^{-k})
    \end{align*}
  \end{block}
\end{frame}

\begin{frame}
  \frametitle{G2(a)}
    \begin{tabular}[]{lr}
\includegraphics[scale=0.15]{g2alp} & \includegraphics[scale=0.15]{g2alk}\\
    \end{tabular}

    \begin{block}{Parameters}
    {\small
      $I_1 = \{(k,l,p)| k,l,p \in \Z, p > -l \}, I_2 = \{(k,l,p)| k,l \in \Z, l \geq 1, p = -l \}$
      }
    \end{block}
    \begin{block}{Eigenvalues}
      $\lambda_{klm}^{\pm} = \pm 2 \pi \sqrt{\frac{(k + 1/2)^2}{H^2} + \frac{l^2}{L^2} + \frac{p^2}{S^2}}$
    \end{block}
    \begin{block}{Transformations}
    $(l,p) \mapsto (p,l)$, $l \mapsto -l$
    \end{block}
\end{frame}

\begin{frame}
  \frametitle{G2(b)(d)}
    \begin{tabular}[]{l}
\includegraphics[scale=0.15]{g2bd}\\
    \end{tabular}
    \begin{block}{}
      Parameters: $\{(k,l,p)| k,l,p \in \Z, l \geq 0\}$

      Eigenvalues: $\pm 2\pi\sqrt{\frac{(k+1/2)^2}{H^2}\frac{(l+1/2)^2}{L^2} + \frac{(p+c+1/2)^2}{S^2}}$
      
      Transformations: $l \mapsto -l - 1$
    \end{block}

\end{frame}

\begin{frame}
\frametitle{G2(c)}
    \begin{tabular}[]{l}
\includegraphics[scale=0.15]{g2c}\\
    \end{tabular}

    \begin{block}{}
      Parameters: $\{(k,l,m)| k,l,m \in \Z, m \geq 0\}$

      Eigenvalues: $\pm 2\pi\sqrt{\frac{(k+1/2)^2}{H^2}\frac{l^2}{L^2} + \frac{(m-1/2)^2}{S^2}}$
      
      Transformations: $l \mapsto -l, p \mapsto 1-p$
    \end{block}
\end{frame}

\begin{frame}
  \frametitle{G3}
    \begin{tabular}[]{l}
\includegraphics[scale=0.15]{g3}\\
    \end{tabular}
    \begin{block}{}
      Parameters: $\{(k,l,m)| k,l,m \in \Z, l \geq 1, 0 \leq m < l\}$

      Eigenvalues: $\pm 2\pi\sqrt{\frac{(k+c)^2}{H^2}\frac{l^2}{L^2} + \frac{(l - 2m)^2}{L^2}}$
      
      Transformations: $(l,m) \mapsto (-l,-m), (l,m) \mapsto (m,l), (l,m) \mapsto(l-m,-m)$
    \end{block}
\end{frame}

\begin{frame}
  \frametitle{G4(a)}
    \begin{tabular}[]{lr}
	\includegraphics[scale=0.15]{g4a}\\
    \end{tabular}
    \begin{block}{}
      Parameters: $\{(k,l,p)| k,l,p \in \Z, l \geq 1, -l \leq p < l\}$

      Eigenvalues: $\pm 2\pi\sqrt{\frac{(k+1/2)^2}{H^2} + \frac{l^2 + p^2}{L^2}}$
      
      Transformations: $l \mapsto -l, (l,p) \mapsto (p,l), (l,p)\mapsto(p,-l)$
    \end{block}
\end{frame}

\begin{frame}
    \frametitle{G4(b)}
    \begin{tabular}[]{l}
	\includegraphics[scale=0.15]{g4b}\\
    \end{tabular}

    \begin{block}{}
      Parameters: $\{(k,l,p)| k,l,p \in \Z, l \geq 1, -l + 1 < p \leq l\}$

      Eigenvalues: $\pm 2\pi\sqrt{\frac{(k+1/2)^2}{H^2} + \frac{(l-1/2)^2 + (p - 1/2)^2}{L^2}}$
      
      Transformations: $l \mapsto -l, (l,p) \mapsto (p,l)$
    \end{block}

\end{frame}

\begin{frame}
\frametitle{G6}
    \begin{tabular}[]{l}
	\includegraphics[scale=0.15]{g6}\\
    \end{tabular}
    \begin{block}{G6}
      Parameters: $\{(k,l,m)| k,l,m \in \Z, l \geq 0, k\geq 0\}$

      Eigenvalues: $\pm 2\pi\sqrt{\frac{(k+1/2)^2}{H^2}\frac{(l+1/2)^2}{L^2} + \frac{(m + 1/2)^2}{S^2}}$
      
      Transformations: $k\mapsto -k - 1, l \mapsto -l -1$
    \end{block}
\end{frame}

\begin{frame}
  \frametitle{G5}
  \begin{block}{Parameters}
    $\{(k,l,m)| k,l,m \in \Z, l \geq 1, 0 \leq m < l\}$
  \end{block}

  \begin{block}{Eigenvalues}
    $\pm 2\pi\sqrt{\frac{(k+1/2)^2}{H^2}\frac{l^2}{L^2} + \frac{(2l- m)^2}{3L^2}}$
  \end{block}

  \pause

  \begin{block}{}
    Surprisingly, the technique doesn't work for $G5$
  \end{block}
\end{frame}

\begin{frame}
\frametitle{Bieberbach manifolds}
\begin{block}{Spectral Action}
\begin{align*}
&\rm{G2(a,b,c,d)} \quad HSL(\frac{\Lambda}{2\pi})^3 \int_{\R ^3} f(u^2 +v^2 + w^2)dudvdw + O(\Lambda^{-\infty})\\
&\rm{G3} \quad HL^2 \frac{1}{\sqrt{3}}(\frac{\Lambda}{2\pi})^3 \int_{\R ^3} f(u^2 +v^2 + w^2)dudvdw + O(\Lambda^{-\infty})\\
&\rm{G4(a,b)} \quad HL^2 \frac{1}{2}(\frac{\Lambda}{2\pi})^3 \int_{\R ^3} f(u^2 +v^2 + w^2)dudvdw + O(\Lambda^{-\infty})\\
&\rm{G6} \quad HLS \frac{1}{2}(\frac{\Lambda}{2\pi})^3 \int_{\R ^3} f(u^2 +v^2 + t^2)dudvdt + O(\Lambda^{-\infty})\\
\end{align*}
\end{block}
\end{frame}

\section{One-parameter family of Dirac operators on $SU(2)$ and $SU(3)$}

\begin{frame}
  \frametitle{Robertson-Walker metric on $SU(2)$}
  \begin{block}{}
    Chamseddine and Connes observed that one may apply the Euler-Maclaurin formula to compute the spectral action for $SU(2)$ equipped with the Robertson-Walker metric.
  \end{block}
\end{frame}

\begin{frame}
  \frametitle{Euler-Maclaurin formula}
  \begin{block}{}
    For a smooth function $g$,
    \begin{align*}
      \sum_{k=a}^b g(k) &= \int_a^b g(x) dx + \frac{g(a) + g(b)}{2} +  \\
      & \sum_{j=2}^m \frac{B_j}{j!}(g^{(j-1)}(b) - g^{(j-1)}(a)) - R_m
    \end{align*}
  \end{block}

  \pause

  \begin{block}{}
    \begin{itemize}
      \item Useful when Dirac spectrum is indexed by $\N$ instead of $\Z$
      \item $B_j$ are the Bernoulli numbers
      \item \[
        R_m = \frac{(-1)^m}{m!} \int_a^b g^{(m)}(x) B_m (x-[x])dx
      \]
    \end{itemize}
  \end{block}
\end{frame}

\begin{frame}
  \frametitle{Bernoulli numbers}
  Bernoulli numbers are value at 0 of Bernoulli polynomials. The Bernoulli polynomials are defined inductively as
  \[
    B_0(x) = 1, B_n'(x) = nB_{n-1}(x), \int_0^1 B_n(x)dx = 0
  \]
\end{frame}

\begin{frame}
  \frametitle{One-parameter family of Dirac operators}
  \begin{block}{One-parameter family of connections}
    \[
      \nabla_X^t = \nabla_X^0 + t[X, \cdot]
    \]
  \end{block}

  \pause

  \begin{block}{}
    \begin{itemize}
      \item Connections are metric
      \item Torsion equals $T(X,Y) = (2t-1)[X,Y]$
      \item Therefore, $t = 1/2$ corresponds to Levi-Civita connection
      \item $t = 1/3$ corresponds to cubic Dirac operator studied by Kostant
    \end{itemize}
  \end{block}
\end{frame}

\begin{frame}
  \begin{block}{Theorem (Lai, Teh)}
    The one-parameter family of Dirac operators $D_t$ may be written in terms of the Casimir operator.
    \[
      1 \otimes \Cas + (3t-1)(\Delta \Cas - 1 \otimes \Cas - \Cas \otimes 1) + 9 t^2 |\rho|^2
    \]
  \end{block}

  \begin{block}{}
    The action of $\Cas$ on irreducible representations is well-understood, and so computing the Dirac spectrum becomes once again, a matter of calculation.
  \end{block}
\end{frame}

\begin{frame}
  \begin{block}{Theorem(Teh)}
	The spectrum of the one-parameter family of Dirac operators on $SU(2)$ may be indexed by $\Z$ such that its multiplicities are 
	\begin{align*}
		& \Tr f( \frac{D_t ^2}{\Lambda ^2}) = \\ 
		&\Lambda ^3 \int_{\R}y^2 f(y^2) dy + \Lambda(3t-1)(3t-2)\int_\R f(y^2) dy + O(\Lambda^{-k})
	\end{align*}
  \end{block}
\end{frame}

\begin{frame}
  \begin{block}{Theorem(Teh)}
    The Dirac operator $D_{1/3}$ on $SU(3)$ has eigenvalues $p^2 + q^2 + pq$ for $(p,q) \in \N^2$ with multiplicities given by $2p^2q^2(p+q)^2$.
  \end{block}

  \begin{block}{}
    \begin{itemize}
      \item The strategy of transforming parameter space to cover $\Z^2$ used for Bieberbach manifolds works here as well.
      \item In this case the linear transformations of parameter space must preserve the multiplicities in addition to the eigenvalues.
    \end{itemize}
  \end{block}
\end{frame}

\begin{frame}
  \begin{block}{Theorem(Teh)}
    For $t = 1/3$, one may apply the Poisson summation formula to compute the spectrum for $SU(3)$, and it is given by:
    \begin{align*}
      & \Tr f(D_{1/3}^2)/\Lambda ^2) = \\
                                   & \frac{1}{3}\iint_{\R ^2} x^2y^2(x+y)^2f(x^2+y^2+xy)dxdy \Lambda^8 + O(\Lambda ^{-k})
    \end{align*}
  \end{block}
\end{frame}

\begin{frame}
  \begin{block}{Two variable Euler-Maclaurin formula}
    \begin{align*}
      &\sum_{p=0}^{\infty} \sum_{q=0}^{\infty} ` g(p,q) =\\
      &L^{2k} \frac{\partial}{\partial h_1}L^{2k} \frac{\partial}{\partial h_2} \int_{h1}^{\infty} \int_{h2}^{\infty} g(p,q)dpdq|_{h1=0, h2 = 0} + R_m(g) \\
    \end{align*}

    Apostrophe indicates weights of 1/2 along boundary edges, 1/4 at corner.

    \[
      L^{2k}(S) = 1 + \frac{1}{2!}B_2S^2+ \ldots + \frac{1}{(2k)!} B_{2k}S^{2k}
    \]

    {\tiny
    \begin{align*}
      &R_m(g) = \\
      & \sum_{I \subsetneq \{1,2\}}(-1)^{(m-1)(2-|I|)}\prod_{i\in I} L^{2k} \frac{\partial}{\partial h_i} \int_{h1}^{\infty}\int_{h1}^{\infty} \prod_{i \notin I} P_m(x_i) \frac{\partial}{\partial x_i} ^m g(x_1,x_2)dx_1 dx_2 |_{h=0}
    \end{align*}
    }
  \end{block}
\end{frame}

\begin{frame}
	\begin{block}{Theorem (Teh)}
		{\tiny
		\begin{align*}
	 		&\Tr f(D_t^2/ \Lambda^2) = \\
			& 2 \iint_{\R ^{+2}} f (x^2 + y^2 + xy)x^2y^2(x+y)^2 dxdy \Lambda^8 +\\
			&3(3t-1)(3t-2)\iint_{\R ^{+2}}f(x^2+y^2+xy)(x^4+2x^3y+3x^2y^2+2xy^3 +y^4)dxdy\Lambda^6 +\\
			&9(3t-1)^2(3t-2)^2\iint_{\R ^{+2}}f(x^2 + y^2+xy)(x^2+xy+y^2)dxdy\Lambda^4 +\\
			&6(3t-1)^3(3t-2)^3 \iint_{\R ^{+2}} f(x^2 + y^2 +xy)dxdy \Lambda^2+ \\
			&O(\Lambda^{-1})
		\end{align*}
		}
	\end{block}
	
	\begin{block}{}
		
	\end{block}
\end{frame}
\end{document}

